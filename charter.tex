\documentclass[11pt]{charter}

\newcommand{\forceindent}{\leavevmode{\parindent=4em\indent}}

% El títulos de la memoria, se usa en la carátula y se puede usar el cualquier lugar del documento con el comando \ttitle
\titulo{Sistema de ensayos de relés ferroviarios de seguridad basado en computación en la nube}

% Nombre del posgrado, se usa en la carátula y se puede usar el cualquier lugar del documento con el comando \degreename
\posgrado{Carrera de Especialización en Sistemas Embebidos}
%\posgrado{Carrera de Especialización en Internet de las Cosas}
%\posgrado{Carrera de Especialización en Intelegencia Artificial}
%\posgrado{Maestría en Sistemas Embebidos}
%\posgrado{Maestría en Internet de las cosas}

% Tu nombre, se puede usar el cualquier lugar del documento con el comando \authorname
\autor{Ing. Gaspar Santamarina}

% El nombre del director y co-director, se puede usar el cualquier lugar del documento con el comando \supname y \cosupname y \pertesupname y \pertecosupname
\director{Ing. Adrián Laiuppa}
\pertenenciaDirector{CONICET-GICSAFe}
% FIXME:NO IMPLEMENTADO EL CODIRECTOR ni su pertenencia
\codirector{Mg. Ing. Martin Menendez} % si queda vacio no se deberíá incluir
\pertenenciaCoDirector{CONICET-GICSAFe}

% Nombre del cliente, quien va a aprobar los resultados del proyecto, se puede usar con el comando \clientename y \empclientename
\cliente{Martín Harris}
\empresaCliente{Trenes Argentinos}

% Nombre y pertenencia de los jurados, se pueden usar el cualquier lugar del documento con el comando \jurunoname, \jurdosname y \jurtresname y \perteunoname, \pertedosname y \pertetresname.
\juradoUno{Nombre y Apellido (1)}
\pertenenciaJurUno{pertenencia (1)}
\juradoDos{Nombre y Apellido (2)}
\pertenenciaJurDos{pertenencia (2)}
\juradoTres{Nombre y Apellido (3)}
\pertenenciaJurTres{pertenencia (3)}

\fechaINICIO{22 de junio de 2020}		%Fecha de inicio de la cursada de GdP \fechaInicioName
\fechaFINALPlanificacion{22 de Agosto de 2020} 	%Fecha de final de cursada de GdP
\fechaFINALTrabajo{22 de junio de 2021}		%Fecha de defensa pública del trabajo final


\begin{document}

\maketitle
\thispagestyle{empty}
\pagebreak


\thispagestyle{empty}
{\setlength{\parskip}{0pt}
\tableofcontents{}
}
\pagebreak


\section{Registros de cambios}
\label{sec:registro}


\begin{table}[ht]
\label{tab:registro}
\centering

\begin{tabularx}{\linewidth}{@{}|c|X|c|@{}}
\hline
\rowcolor[HTML]{C0C0C0} 
Revisión & \multicolumn{1}{c|}{\cellcolor[HTML]{C0C0C0}Detalles de los cambios realizados} & Fecha      \\ \hline
1.0      & Creación del documento: propósito, alcance, supuestos\newline                                                        y tareas & 09/07/2020 \\ \hline
1.1      & Se agrega AoN, diagrama Gantt, matriz de recursos,\newline
presupuesto y matriz de responsabilidades & 30/07/2020 \\ \hline
\end{tabularx}
\end{table}

\pagebreak



\section{Acta de constitución del proyecto}
\label{sec:acta}

\begin{flushright}
Buenos Aires, \fechaInicioName
\end{flushright}

\vspace{2cm}

Por medio de la presente se acuerda con el Ing. \authorname\hspace{1px} que su Trabajo Final de la \degreename\hspace{1px} se titulará ``\ttitle'', consistirá esencialmente en el prototipo preliminar de un probador de relés ferroviarios de seguridad, basado en hardware digital, con la posibilidad de ser operado de forma remota, y tendrá un presupuesto preliminar estimado de 687 hs de trabajo, con fecha de inicio \fechaInicioName\hspace{1px} y fecha de presentación pública \fechaFinalName.

Se adjunta a esta acta la planificación inicial.

\vfill

% Esta parte se construye sola con la información que hayan cargado en el preámbulo del documento y no debe modificarla
\begin{table}[ht]
\centering
\begin{tabular}{ccc}
\begin{tabular}[c]{@{}c@{}}Ariel Lutenberg \\ Director posgrado FIUBA\end{tabular} &  & \begin{tabular}[c]{@{}c@{}}\clientename \\ \empclientename \end{tabular} \vspace{2.5cm} \\ 
\multicolumn{3}{c}{\begin{tabular}[c]{@{}c@{}} \supname \\ Director del Trabajo Final\end{tabular}} \vspace{2.5cm} \\
\begin{tabular}[c]{@{}c@{}}\jurunoname \\ Jurado del Trabajo Final\end{tabular}     &  & \begin{tabular}[c]{@{}c@{}}\jurdosname\\ Jurado del Trabajo Final\end{tabular}  \vspace{2.5cm}  \\
\multicolumn{3}{c}{\begin{tabular}[c]{@{}c@{}} \jurtresname\\ Jurado del Trabajo Final\end{tabular}} \vspace{.5cm}                                                                     
\end{tabular}
\end{table}




\section{Descripción técnica-conceptual del Proyecto a realizar}
\label{sec:descripcion}

Las barreras automáticas de los pasos a nivel y los sistemas de cambios de vía del sistema ferroviario de la Argentina dependen mayormente de componentes  electromecánicos. Estos componentes deben cumplir altos niveles de seguridad para alcanzar la fiabilidad necesaria. Un componente importante de estos sistemas son los relés de señalamiento, llamados también relés de seguridad (“safety relays” en inglés) o relés vitales (“vital relays” en inglés).

Un único paso a nivel automático puede emplear decenas de relés. Estos solo se consiguen por importación y cada uno tiene un valor superior a los U\$S 1,000 (mil dólares estadounidenses). Es conveniente entonces el desarrollo de una industria nacional que fabrique relés certificados; un sistema de ensayos de relés es una parte esencial para elaborar una certificación local.

El diseño del sistema de ensayos de relés se basa en la norma UNE-EN 50578. En la misma se establecen valores máximos y mínimos en las variaciones de los valores iniciales de la corriente de excitación, la corriente de caída y el factor K. El sistema de ensayos de relés debe permitir el monitoreo de estos valores eléctricos.

En la Figura \ref{fig:diagBloques2} se puede ver la arquitectura completa del sistema, incluyendo el servidor remoto encargado de monitorear el ensayo a lo largo de todo el proceso.

\vspace{25px}

\begin{figure}[H]
\centering 
\includegraphics[width=.6\textwidth]{./Figuras/bloques.png}
\caption{Diagrama en bloques del sistema}
\label{fig:diagBloques2}
\end{figure}

\vspace{25px}

El firmware correrá sobre 3 placas CIAA-NXP en simultáneo para asegurar los niveles de disponibilidad deseados. Sobre cada una de las CIAA-NXP se monta una placa de lógica no programable. Este subsistema cuenta entre otras con las siguientes funciones:

\begin{itemize}
\item Generar la señal de activación del relé (incluyendo los transistores de potencia y circuitos de activación correspondientes).
\item Realizar la sincronización automática de las señales de activación del relé generadas por tres placas independientes entre sí.
\item Detectar cuando una de las tres placas falla y en ese caso desconectarla del relé en forma automática.
\item Realizar la medición de múltiples valores de voltaje y corriente asociados al funcionamiento del relé. Estos valores son el voltaje aplicado al bobinado del relé y la corriente que circula, junto con los valores de voltaje aplicado y la corriente que circula por cada contacto. Se ensayarán relés que usualmente cuentan con hasta seis contactos normal cerrado y seis contactos normal abierto.
\item Mostrar indicaciones luminosas relativas al estado del sistema mediante diodos LED.
\item Establecer una comunicación MQTT sobre TCP/IP con el broker para recibir los comandos del usuario y enviar, en tiempo real, los datos recabados.
\end{itemize}

Para cumplir con los requerimientos de la norma, el sistema debe ser capaz de realizar los siguientes ensayos:

\begin{itemize}
\item \textbf{Sistema magnético, corriente de excitación y corriente de caída:} la corriente de excitación se define como la corriente mínima a través de la bobina que, partiendo de un valor nulo, alcanza para mover la armadura de la posición de reposo a la posición de trabajo y para aplicar la fuerza de contacto especificada por el fabricante, cerrando todos los contactos de trabajo. La corriente de caída es la corriente máxima a través de la bobina que, partiendo del valor de la corriente nominal, produce la apertura de todos los contactos de trabajo. Este ensayo permite medir el factor K. Para esto se debe generar una función de rampa que, luego de ser amplificada, se aplica al bobinado del relé.
\item \textbf{Ensayo de vida útil mecánica:} este ensayo es en vacío. El usuario puede ingresar la cantidad de ciclos a ensayar siendo el valor máximo de $10 \cdot 10^6$ movimientos porque la norma se cumple cuando se llega esta cantidad de ciclos. Este ensayo registra la corriente y el voltaje del bobinado y el estado de los contactos.
\item \textbf{Ensayo con carga:} este ensayo se realiza aplicando el voltaje nominal los contactos abiertos y la corriente nominal a los contactos cerrados. A la bobina del relé también se le aplica el voltaje nominal. La norma dice que en estas condiciones se debe asegurar una cantidad mínima de $2 \cdot 10^6$ movimientos durante la vida útil del relé. En este ensayo se registran los voltajes y las corrientes de la bobina y los voltajes y las corrientes de los contactos.
\end{itemize}

\section{Identificación y análisis de los interesados}
\label{sec:interesados}

\begin{table}[H]
%\caption{Identificación de los interesados}
%\label{tab:interesados}
\begin{tabularx}{\linewidth}{@{}|l|X|X|l|@{}}
\hline
\rowcolor[HTML]{C0C0C0} 
Rol           & Nombre y Apellido & Organización 	  & Puesto 	\\ \hline
Cliente       & \clientename      &\empclientename	  & \shortstack {Coordinador General\\de Desarrollo,\\de la Subgerencia\\de Desarrollo y\\Normas Técnicas}\\ \hline
Responsable   & \authorname       & CESE 12Co2020 	  & Alumno 	\\ \hline
Orientador    & \supname	       & \pertesupname 	  & Director	Trabajo final \\ \hline
\multirow{2}{*}{Colaboradores} & Ing. Nicolás Locatelli & CONICET-GICSAFe & \shortstack[l]{Integrante} \\ \cline{2-4}
& Ing. Gustavo Ramoscelli & CONICET-GICSAFe & \shortstack[l]{Investigador} \\ \hline
\end{tabularx}
\textbf{CONICET-GICSAFe:} Grupo de Investigación en Calidad y Seguridad de las Aplicaciones Ferroviarias
\end{table}

\section{1. Propósito del proyecto}
\label{sec:proposito}

El propósito de este proyecto es desarrollar el firmware para un probador de relés de seguridad ferroviarios. El mismo deberá comunicarse con un servidor remoto, desde el cuál se configurarán e iniciarán las pruebas. Además, deberá reportar en tiempo real el estado de las pruebas, junto con las mediciones que la conforman.

\section{2. Alcance del proyecto}
\label{sec:alcance}

Se diseñará e implementará el firmware capaz de realizar, junto con el hardware, las tres pruebas mencionadas anteriormente en la descripción técnica del proyecto. 
En la entrega final no se incluirá el desarrollo de hardware del probador ni el software del servidor remoto.

\section{3. Supuestos del proyecto}
\label{sec:supuestos}

\begin{itemize}
\item Se contará con el hardware terminado al arrancar con el desarrollo de firmware.
\item Se dispondrá, en una etapa avanzada del desarrollo, de un servidor de pruebas con las funcionalidades mínimas requeridas por el probador.
\end{itemize}

\section{4. Requerimientos}
\label{sec:requerimientos}

\begin{enumerate}
\item Interfaces Externas
	\begin{enumerate}
	\item El sistema deberá ser capaz de leer el estado de 5 señales digitales.
	\item El sistema deberá leer 15 señales analógicas a través de 2 integrados ADC externos.
	\item El sistema deberá leer 2 señales analógicas a través de 2 ADC internos.
	\item El sistema deberá leer el estado de un pin digital.
	\item El sistema deberá controlar el estado de 2 pines digitales.
	\item El sistema deberá comunicarse con la tarjeta MicroSD presente en el hardware.
	\item El sistema deberá hacer uso de un puerto Ethernet para conectarse a Internet.
	\item El sistema deberá ser capaz de generar una rampa de tensión configurable.
	\end{enumerate}
\item Funciones
	\begin{enumerate}
	\item El sistema debe ser capaz de enviar notificaciones de alerta al servidor remoto, generadas por el disparo de una señal de alerta del hardware.
	\item El sistema debe ser capaz de enviar notificaciones de alerta al servidor remoto, generadas por sobretensión en cualquiera de las entradas analógicas.
	\item El sistema debe ser capaz de interpretar los siguientes comandos enviados desde el servidor:
		\begin{enumerate}
		\item Aplicar configuración de sistema.
		\item Definir los parámetros asociados a cada tipo de ensayo.
		\item Disparar ensayos.
		\item Parar ensayos.
		\end{enumerate}
	\item El sistema deberá ser capaz de ejecutar los siguientes ensayos:
		\begin{enumerate}
		\item \textbf{Sistema magnético, corriente de excitación y corriente de caída:} Se genera una rampa de tensión de pendiente positiva y luego otra de pendiente negativa. El valor inicial, valor final, paso de tensión y duración del paso de las señales deben ser configurables en ambos casos, además de la máxima corriente de excitación admitida. Se debe notificar al servidor entre paso y paso, indicando la tensión instantánea de la rampa y el valor de las 17 señales analógicas. La prueba finaliza al completarse la cantidad de repeticiones especificadas al inicio de la prueba.
		\item \textbf{Ensayo de vida útil mecánica con y sin carga:} Se debe especificar el número de ciclos. Al detectarse la secuencia de inicio se comienzan a contabilizar los ciclos, representados por el periodo entre cada secuencia de inicio. Se debe notificar al servidor cada vez que ocurre un flanco positivo de las señales CLOCK\_1S o CLOCK\_3S, indicando el número de ciclo y el valor de las 17 señales analógicas. La prueba finaliza al completarse la cantidad de ciclos especificados al inicio de la prueba.
		\end{enumerate}
	\item Junto con cada paquete de datos generado a partir de una prueba, se debe anexar un ID que identifique el tipo de prueba.
	\item Junto con cada paquete de datos se debe anexar una marca temporal y un ID que identifique a la placa dentro de las tres placas que conforman el sistema.
	\item Cada paquete de datos debe ser almacenado en una tarjeta MicroSD a modo de copia de seguridad.
	\item El sistema debe permitir realizar distintos tipos de ensayos en forma intercalada.
	\item El sistema debe permitir configurar los siguientes parámetros:
		\begin{enumerate}
		\item Hora del sistema.
		\item URL y puerto del broker MQTT.
		\end{enumerate}
	\end{enumerate}
\item Requisitos de Rendimiento
	\begin{enumerate}
	\item Para el ensayo de rampa se deben permitir hasta 25 ciclos.
	\item Para el ensayo de vida útil mecánica \textbf{sin carga} se deben permitir hasta $10 \cdot 10^6$ ciclos.
	\item Para el ensayo de vida útil mecánica \textbf{con carga} se deben permitir hasta $2 \cdot 10^6$ ciclos.
	\item Para el ensayo de vida útil mecánica el periodo máximo entre ciclos es de 3 segundos.
	\end{enumerate}
\item Restricciones de Diseño
	\begin{enumerate}
	\item Se usará I$^2$C como protocolo de comunicación con los ADC.
	\item Se usará SPI como protocolo de comunicación con la SD.
	\item Se usará MQTT sobre TCP/IP como protocolo de comunicación con el servidor remoto.
	\item Se usará JSON como formato de texto para intercambio de datos con el servidor remoto.
	\item La rampa de tensión debe ser capaz de alcanzar un valor máximo de hasta 10V.
	\item Se usará la plataforma CIAA-NXP como placa de desarrollo.
	\item Se usará un RTC para mantener la hora de forma local.
	\item Se usará ``Unix time'' como formato de marca temporal. 
	\end{enumerate}
\item Atributos del Sistema
	\begin{enumerate}
	\item El sistema debe ser capaz de funcionar de forma ininterrumpida durante 2 años.
	\item El sistema debe hacer uso de cifrado TLS para la capa de transporte.
	\item El sistema debe hacer uso de Usuario/Contraseña para autenticarse con el broker MQTT.
	\item El desarrollo de software debe seguir una metodología acorde a la norma UNE-EN 50128.
	\end{enumerate}
\item Otros Requisitos
	\begin{enumerate}
	\item El desarrollo debe ser documentado utilizando la herramienta Doxygen.
	\end{enumerate}
\end{enumerate}

%\subsubsection*{4.1 Especificación de formato de archivos JSON}
%\label{sec:espec_json}
%
%A continuación se detallan los pares nombre/valor para los distintos archivos JSON que se intercambiarán entre firmware y servidor. Algunos son comunes a todos los mensajes mientras que otros dependen del comando en cuestión.
%
%\textbf{Pares nombre/valor comunes a todos los JSON:}
%
%\textbf{``board\_id''}\textit{(integer)}: Número de placa.\\
%\textbf{``cmd''}\textit{(string)}: Cadena que identifica al comando. Los posibles comandos son: \\
%	\forceindent test\_config = Configuración del ensayo.\\
%	\forceindent test\_start = Disparo del ensayo.\\
%	\forceindent test\_stop = Parada del ensayo.\\
%\textbf{``params''}\textit{(object)}: Dentro de este objeto se anidan los parámetros asociados al comando. \\
%
%\textbf{Parámetros específicos:}
%
%\underline{\Large test\_config:}\\
%\textbf{``type''}\textit{(integer)}: Tipo de ensayo.\\
%	\forceindent 1 = Sistema magnético, corriente de excitación y corriente de caída.\\
%	\forceindent 2 = Ensayo de vida útil mecánica sin carga.\\
%	\forceindent 3 = Ensayo de vida útil mecánica con carga.\\
%\textbf{``cycles''}\textit{(integer)}: Número de ciclos a repetir.\\
%{\large - Parámetros específicos del ensayo tipo 1:}\\
%\textbf{``vem''}\textit{(integer)}: Tensión máxima de la rampa [V].\\
%\textbf{``inom''}\textit{(integer)}: Corriente nominal [mA].\\
%\textbf{``step''}\textit{(integer)}: Salto de voltaje [mV].\\
%\textbf{``dt''}\textit{(integer)}: Duración del paso [ms].\\
%\textbf{``ip\_max''}\textit{(integer)}: Máxima corriente de excitación admitida [mA].\\
%
%\underline{\Large test\_start:}\\
%\textbf{``type''}\textit{(integer)}: Tipo de ensayo.\\


\section{5. Entregables principales del proyecto}
\label{sec:entregables}

\begin{itemize}
\item Código fuente del firmware
\item Informe de avance
\item Memoria del trabajo
\end{itemize}

\section{6. Desglose del trabajo en tareas}
\label{sec:wbs}

\begin{enumerate}
\item Planificación \hfill (55 hs)
	\begin{enumerate}
	\item Estudio del funcionamiento básico de un probador. \hfill (25 hs)
	\item Elaboración del documento de Planificación del proyecto. \hfill (30 hs)
	\end{enumerate}
\item Pruebas preliminares \hfill (80 hs)
	\begin{enumerate}
	\item Desarrollo del prototipo de firmware. \hfill (25 hs)
	\item Pruebas de interacción con el hardware. \hfill (35 hs)
	\item Pruebas de comunicación con el servidor. \hfill (20 hs)
	\end{enumerate}
\item Diseño de firmware \hfill (100 hs)
	\begin{enumerate}
	\item Diseño de la arquitectura general y el flujo de datos. \hfill (35 hs)
	\item Diseño de las tareas del RTOS y los mecanismos \newline de comunicación entre ellas.  \hfill (35 hs)
	\item Configuración del entorno de desarrollo. \hfill (15 hs)
	\item Estudio de las herramientas de software disponibles para la plataforma \newline de desarrollo elegida. \hfill (15 hs)
	\end{enumerate}
\item Implementación de firmware \hfill (229 hs)
	\begin{enumerate}
	\item Desarrollo del driver para los ADC. \hfill (25 hs)
	\item Desarrollo del módulo de adquisición. \hfill (32 hs)
	\item Desarrollo del driver ethernet. \hfill (40 hs)
	\item Desarrollo del módulo de comunicación. \hfill (40 hs)
	\item Desarrollo del módulo de almacenamiento SD.  \hfill (32 hs)
	\item Desarrollo del módulo de detección de estados. \hfill (20 hs)
	\item Desarrollo del módulo de control principal. \hfill (40 hs)
	\end{enumerate}
\item Verificación y Validación \hfill (163 hs)
	\begin{enumerate}
	\item Pruebas individuales de los módulos de firmware.  \hfill (30 hs)
	\item Ensayos de integración del firmware.  \hfill (20 hs)
	\item Pruebas de integración con el hardware. \hfill (20 hs)
	\item Ensayos funcionales. \hfill (40 hs)
	\item Análisis de los efectos de los errores de software. \hfill (20 hs)
	\item Demostración formal. \hfill (8 hs)
	\item Elaboración del informe de validación. \hfill (25 hs)
	\end{enumerate}
\item Documentación \hfill (60 hs)
	\begin{enumerate}
	\item Elaboración de la Memoria del trabajo. \hfill (40 hs)
	\item Producción de la presentación final. \hfill (20 hs)
	\end{enumerate}
\end{enumerate}

Cantidad total de horas: (687 hs)

\section{7. Diagrama de Activity On Node}
\label{sec:AoN}

\begin{figure}[H]
\centering 
\includegraphics[width=.8\textwidth]{./Figuras/AoN2.png}
\caption{Diagrama en \textit{Activity on Node}}
\label{fig:AoN}
\end{figure}

\section{8. Diagrama de Gantt}
\label{sec:gantt}

\begin{center}
\includegraphics[width=0.95\textwidth, angle=180]{./Figuras/gantt-0.jpg}
\captionof{figure}{Diagrama de gantt}
\label{fig:gantt}
\end{center}

\begin{center}
\includegraphics[width=0.95\textwidth, angle=180]{./Figuras/gantt-1.jpg}
\captionof{figure}{Diagrama de gantt (continuación)}
\label{fig:gantt}
\end{center}

\section{9. Matriz de uso de recursos de materiales}
\label{sec:recursos}

\begin{table}[H]
\label{tab:recursos}
\centering
\begin{tabularx}{\linewidth}{@{}|c|X|c|c|@{}}
\hline
\cellcolor[HTML]{C0C0C0} & \cellcolor[HTML]{C0C0C0} & \multicolumn{2}{c|}{\cellcolor[HTML]{C0C0C0}Recursos requeridos (horas)} \\ \cline{3-4} 
\multirow{-2}{*}{\cellcolor[HTML]{C0C0C0}WBS} & \multirow{-2}{*}{\cellcolor[HTML]{C0C0C0}\begin{tabular}[c]{@{}c@{}}Nombre de la tarea\end{tabular}} & PC & Probador+CIAA \\ \hline
 1.1 & Estudio del funcionamiento básico de un probador & 25 & 10 \\ \hline
 1.2 & Elaboración del documento de planificación del proyecto & 30 & \\ \hline
 2.1 & Desarrollo del prototipo de firmware & 25 & 10 \\ \hline
 2.2 & Pruebas de interacción con el hardware & 35 & 10\\ \hline
 2.3 & Pruebas de comunicación con el servidor & 20 & \\ \hline
 3.1 & Diseño de la arquitectura general y el flujo de datos & 35 & \\ \hline
 3.2 & Diseño de las tareas del RTOS y los mecanismos de comunicación entre ellas & 35 & \\ \hline
 3.3 & Configuración del entorno de desarrollo & 15 & \\ \hline
 3.4 & Estudio de las herrameintas de software disponible para la plataforma de desarrollo elegida & 15 & \\ \hline
 4.1 & Desarrollo del driver para los ADC & 25 & 15 \\ \hline
 4.2 & Desarrollo del módulo de adquisición & 32 & 5 \\ \hline
 4.3 & Desarrollo del driver ethernet & 40 & 5 \\ \hline
 4.4 & Desarrollo del módulo de comunicación & 40 & \\ \hline
 4.5 & Desarrollo del módulo de almacenamiento SD & 32 & 5 \\ \hline
 4.6 & Desarrollo del módulo de detección de estados & 20 & 5 \\ \hline
 4.7 & Desarrollo del módulo de control principal & 40 & \\ \hline
 5.1 & Pruebas individuales de los módulos de firmware & 30 & 10 \\ \hline
 5.2 & Ensayos de integración del firmware & 20 & 20 \\ \hline
 5.3 & Pruebas de integración con el hardware & 20 & 20 \\ \hline
 5.4 & Ensayos funcionales & 40 & 40 \\ \hline
 5.5 & Análisis de los efectos de los errores de software 20 & 15 & \\ \hline
 5.6 & Demostración formal & 8 & 8 \\ \hline
 5.7 & Elaboración del informe de validación & 25 & \\ \hline 
 6.1 & Elaboración de la memoria de trabajo & 40 & \\ \hline 
 6.2 & Producción de la presentación final & 20 & \\ \hline 
\end{tabularx}%
\end{table}


\section{10. Presupuesto detallado del proyecto}
\label{sec:presupuesto}

\begin{table}[H]
\centering
\begin{tabularx}{\linewidth}{@{}|X|c|r|r|@{}}
\hline
\rowcolor[HTML]{C0C0C0} 
\multicolumn{4}{|c|}{\cellcolor[HTML]{C0C0C0}COSTOS DIRECTOS} \\ \hline
\rowcolor[HTML]{C0C0C0} 
Descripción &
  \multicolumn{1}{c|}{\cellcolor[HTML]{C0C0C0}Cantidad} &
  \multicolumn{1}{c|}{\cellcolor[HTML]{C0C0C0}Valor unitario} &
  \multicolumn{1}{c|}{\cellcolor[HTML]{C0C0C0}Valor total} \\ \hline
CIAA-NXP                 & 1        & \$ 25.600  & \$ 25.600  \\  \hline
Poncho Probador de relés & 1        & \$ 12.693        & \$ 12.693     \\ \hline
Mano de obra             & 687 hs   & \$ 850     & \$ 583.950 \\ \hline
\multicolumn{3}{|c|}{SUBTOTAL} &
  \multicolumn{1}{c|}{\$ 622.243} \\ \hline
\rowcolor[HTML]{C0C0C0} 
\multicolumn{4}{|c|}{\cellcolor[HTML]{C0C0C0}COSTOS INDIRECTOS} \\ \hline
\rowcolor[HTML]{C0C0C0}
Descripción &
  \multicolumn{1}{c|}{\cellcolor[HTML]{C0C0C0}Cantidad} &
  \multicolumn{1}{c|}{\cellcolor[HTML]{C0C0C0}Valor unitario} &
  \multicolumn{1}{c|}{\cellcolor[HTML]{C0C0C0}Valor total} \\ \hline
30\% de los costos directos & N/A & N/A & \$ 186.673  \\  \hline
\multicolumn{3}{|c|}{SUBTOTAL} &
  \multicolumn{1}{c|}{} \\ \hline
\rowcolor[HTML]{C0C0C0}
\multicolumn{3}{|c|}{TOTAL} &
   \\ \hline
\end{tabularx}%
\end{table}


\section{11. Matriz de asignación de responsabilidades}
\label{sec:responsabilidades}

\begin{table}[htpb]
\centering
\resizebox{\textwidth}{!}{%
\begin{tabular}{|c|m{6cm}|c|c|c|c|c|}
\hline
\rowcolor[HTML]{C0C0C0} 
\cellcolor[HTML]{C0C0C0} &
  \cellcolor[HTML]{C0C0C0} &
  \multicolumn{5}{c|}{\cellcolor[HTML]{C0C0C0}Nombres y roles del proyecto} \\ \cline{3-6} 
\rowcolor[HTML]{C0C0C0} 
\cellcolor[HTML]{C0C0C0} &
  \cellcolor[HTML]{C0C0C0} &
  Responsable &
  Orientador &
  Colaborador &
  Colaborador &
  Cliente \\ \cline{3-6} 
\rowcolor[HTML]{C0C0C0} 
\multirow{-3}{*}{\cellcolor[HTML]{C0C0C0}\begin{tabular}[c]{@{}c@{}}Código\\ WBS\end{tabular}} &
  \multirow{-3}{*}{\cellcolor[HTML]{C0C0C0}Nombre de la tarea} &
  \authorname &
  \supname &
  Ing. Nicolás Locatelli &
  Ing. Gustavo Ramoscelli &
  \clientename \\ \hline
 1.1 & Estudio del funcionamiento básico de un probador & P & C & & & \\ \hline
 1.2 & Elaboración del documento de planificación del proyecto & P & A & & & I \\ \hline
 2.1 & Desarrollo del prototipo de firmware & P & C & & & \\ \hline
 2.2 & Pruebas de interacción con el hardware & P & C & & & \\ \hline
 2.3 & Pruebas de comunicación con el servidor & P & & C & C & \\ \hline
 3.1 & Diseño de la arquitectura general y el flujo de datos & P & & & & \\ \hline
 3.2 & Diseño de las tareas del RTOS y los mecanismos de comunicación entre ellas & P & & & & \\ \hline
 3.3 & Configuración del entorno de desarrollo & P & & & & \\ \hline
 3.4 & Estudio de las herrameintas de software disponible para la plataforma de desarrollo elegida & P & & & & \\ \hline
 4.1 & Desarrollo del driver para los ADC & P & C & & & \\ \hline
 4.2 & Desarrollo del módulo de adquisición & P & I & & & \\ \hline
 4.3 & Desarrollo del driver ethernet & P & & & & \\ \hline
 4.4 & Desarrollo del módulo de comunicación & P & & I & I & \\ \hline
 4.5 & Desarrollo del módulo de almacenamiento SD & P & I & & & \\ \hline
 4.6 & Desarrollo del módulo de detección de estados & P & C & & & \\ \hline
 4.7 & Desarrollo del módulo de control principal & P & I & & & \\ \hline
 5.1 & Pruebas individuales de los módulos de firmware & P & & & & \\ \hline
 5.2 & Ensayos de integración del firmware & P & I & & & \\ \hline
 5.3 & Pruebas de integración con el hardware & P & A & & & \\ \hline
 5.4 & Ensayos funcionales & P & A & A & A & A \\ \hline
 5.5 & Análisis de los efectos de los errores de software & P & C & & & \\ \hline
 5.6 & Demostración formal & P & A & I & I & A \\ \hline
 5.7 & Elaboración del informe de validación & P & A & & & \\ \hline 
 6.1 & Elaboración de la memoria de trabajo & P & A & & & \\ \hline 
 6.2 & Producción de la presentación final & P & A & & & \\ \hline 
\end{tabular}%
}
\end{table}

{\footnotesize
Referencias:
\begin{itemize}
	\item P = Responsabilidad Primaria
	\item S = Responsabilidad Secundaria
	\item A = Aprobación
	\item I = Informado
	\item C = Consultado
\end{itemize}
} %footnotesize

\section{12. Gestión de riesgos}
\label{sec:riesgos}

\begin{consigna}{red}
a) Identificación de los riesgos (al menos cinco) y estimación de sus consecuencias:
 
Riesgo 1: detallar el riesgo (riesgo es algo que si ocurre altera los planes previstos)
\begin{itemize}
\item Severidad (S): mientras más severo, más alto es el número (usar números del 1 al 10).\\
Justificar el motivo por el cual se asigna determinado número de severidad (S).
\item Probabilidad de ocurrencia (O): mientras más probable, más alto es el número (usar del 1 al 10).\\
Justificar el motivo por el cual se asigna determinado número de (O). 
\end{itemize}   

Riesgo 2:
\begin{itemize}
\item Severidad (S): 
\item Ocurrencia (O):
\end{itemize}

Riesgo 3:
\begin{itemize}
\item Severidad (S): 
\item Ocurrencia (O):
\end{itemize}


b) Tabla de gestión de riesgos:      (El RPN se calcula como RPN=SxO)

\begin{table}[htpb]
\centering
\begin{tabularx}{\linewidth}{@{}|X|c|c|c|c|c|c|@{}}
\hline
\rowcolor[HTML]{C0C0C0} 
Riesgo & S & O & RPN & S* & O* & RPN* \\ \hline
       &   &   &     &    &    &      \\ \hline
       &   &   &     &    &    &      \\ \hline
       &   &   &     &    &    &      \\ \hline
       &   &   &     &    &    &      \\ \hline
       &   &   &     &    &    &      \\ \hline
\end{tabularx}%
\end{table}

Criterio adoptado: 
Se tomarán medidas de mitigación en los riesgos cuyos números de RPN sean mayores a ....

Nota: los valores marcados con (*) en la tabla corresponden luego de haber aplicado la mitigación.

c) Plan de mitigación de los riesgos que originalmente excedían el RPN máximo establecido:
 
Riesgo 1: Plan de mitigación (si por el RPN fuera necesario elaborar un plan de mitigación).
  Nueva asignación de S y O, con su respectiva justificación:
  - Severidad (S): mientras más severo, más alto es el número (usar números del 1 al 10).
          Justificar el motivo por el cual se asigna determinado número de severidad (S).
  - Probabilidad de ocurrencia (O): mientras más probable, más alto es el número (usar del 1 al 10).
          Justificar el motivo por el cual se asigna determinado número de (O).

Riesgo 2: Plan de mitigación (si por el RPN fuera necesario elaborar un plan de mitigación).
 
Riesgo 3: Plan de mitigación (si por el RPN fuera necesario elaborar un plan de mitigación)

\end{consigna}


\section{13. Gestión de la calidad}
\label{sec:calidad}

\begin{consigna}{red}
Para cada uno de los requerimientos del proyecto indique:
\begin{itemize} 
\item Req \#1: Copiar acá el requerimiento.

Verificación y validación:

\begin{itemize}
\item Verificación para confirmar si se cumplió con lo requerido antes de mostrar el sistema al cliente:\\
Detallar 
\item Validación con el cliente para confirmar que está de acuerdo en que se cumplió con lo requerido:\\
Detallar  
\end{itemize}

\end{itemize}

Tener en cuenta que en este contexto se pueden mencionar simulaciones, cálculos, revisión de hojas de datos, consulta con expertos, etc.

\end{consigna}

\section{14. Comunicación del proyecto}
\label{sec:comunicaciones}

\begin{consigna}{red}
El plan de comunicación del proyecto es el siguiente:
\end{consigna}

% Please add the following required packages to your document preamble:
% \usepackage{graphicx}
% \usepackage[table,xcdraw]{xcolor}
% If you use beamer only pass "xcolor=table" option, i.e. \documentclass[xcolor=table]{beamer}
\begin{table}[htpb]
\centering
\resizebox{\textwidth}{!}{%
\begin{tabular}{|c|c|c|c|c|c|}
\hline
\rowcolor[HTML]{C0C0C0} 
\multicolumn{6}{|c|}{\cellcolor[HTML]{C0C0C0}PLAN DE COMUNICACIÓN DEL PROYECTO}           \\ \hline
\rowcolor[HTML]{C0C0C0} 
¿Qué comunicar? & Audiencia & Propósito & Frecuencia & Método de comunicac. & Responsable \\ \hline
                &           &           &            &                      &             \\ \hline
                &           &           &            &                      &             \\ \hline
                &           &           &            &                      &             \\ \hline
                &           &           &            &                      &             \\ \hline
                &           &           &            &                      &             \\ \hline
\end{tabular}%
}
\end{table}

\section{15. Gestión de Compras}
\label{sec:compras}

\begin{consigna}{red}
En caso de tener que comprar elementos o contratar servicios:
a) Explique con qué criterios elegiría a un proveedor.
b) Redacte el Statement of Work correspondiente.
\end{consigna}

\section{16. Seguimiento y control}
\label{sec:seguimiento}

\begin{consigna}{red}
Para cada tarea del proyecto establecer la frecuencia y los indicadores con los se seguirá su avance y quién será el responsable de hacer dicho seguimiento y a quién debe comunicarse la situación (en concordancia con el Plan de Comunicación del proyecto).

El indicador de avance tiene que ser algo medible, mejor incluso si se puede medir en \% de avance. Por ejemplo,se pueden indicar en esta columna cosas como ``cantidad de conexiones ruteadeas'' o ``cantidad de funciones implementadas'', pero no algo genérico y ambiguo como ``\%'', porque el lector no sabe porcentaje de qué cosa.

\end{consigna}

\begin{table}[!htpb]
\centering
\begin{tabularx}{\linewidth}{@{}|X|X|X|X|X|X|@{}}
\hline
\rowcolor[HTML]{C0C0C0} 
\multicolumn{6}{|c|}{\cellcolor[HTML]{C0C0C0}SEGUIMIENTO DE AVANCE}                                                                       \\ \hline
\rowcolor[HTML]{C0C0C0} 
Tarea del WBS & Indicador de avance & Frecuencia de reporte & Resp. de seguimiento & Persona a ser informada & Método de comunic. \\ \hline
 &  &  &  &  &  \\ \hline
 &  &  &  &  &  \\ \hline
 &  &  &  &  &  \\ \hline
 &  &  &  &  &  \\ \hline
 &  &  &  &  &  \\ \hline
\end{tabularx}%
%}
\end{table}

\section{17. Procesos de cierre}    
\label{sec:cierre}

\begin{consigna}{red}
Establecer las pautas de trabajo para realizar una reunión final de evaluación del proyecto, tal que contemple las siguientes actividades:

\begin{itemize}
\item Pautas de trabajo que se seguirán para analizar si se respetó el Plan de Proyecto original:
 - Indicar quién se ocupará de hacer esto y cuál será el procedimiento a aplicar. 
\item Identificación de las técnicas y procedimientos útiles e inútiles que se utilizaron, y los problemas que surgieron y cómo se solucionaron:
 - Indicar quién se ocupará de hacer esto y cuál será el procedimiento para dejar registro.
\item Indicar quién organizará el acto de agradecimiento a todos los interesados, y en especial al equipo de trabajo y colaboradores:
  - Indicar esto y quién financiará los gastos correspondientes.
\end{itemize}

\end{consigna}


\end{document}
